%% Based on a TeXnicCenter-Template by Tino Weinkauf.
%%%%%%%%%%%%%%%%%%%%%%%%%%%%%%%%%%%%%%%%%%%%%%%%%%%%%%%%%%%%%

%%%%%%%%%%%%%%%%%%%%%%%%%%%%%%%%%%%%%%%%%%%%%%%%%%%%%%%%%%%%%
%% HEADER
%%%%%%%%%%%%%%%%%%%%%%%%%%%%%%%%%%%%%%%%%%%%%%%%%%%%%%%%%%%%%
\documentclass[12pt]{report}
% Alternative Options:
%	Paper Size: a4paper / a5paper / b5paper / letterpaper / legalpaper / executivepaper
% Duplex: oneside / twoside
% Base Font Size: 10pt / 11pt / 12pt
\setlength{\parskip}{1em}

%% Packages for Graphics & Figures %%%%%%%%%%%%%%%%%%%%%%%%%%
\usepackage{graphicx} %%For loading graphic files
%\usepackage{subfig} %%Subfigures inside a figure
%\usepackage{pst-all} %%PSTricks - not useable with pdfLaTeX


%% Math Packages %%%%%%%%%%%%%%%%%%%%%%%%%%%%%%%%%%%%%%%%%%%%
\usepackage{amsmath}
\usepackage{amsthm}
\usepackage{amsfonts}
\usepackage{caption}
\usepackage{subcaption}
\newtheorem{theorem}{Theorem}


%% Line Spacing %%%%%%%%%%%%%%%%%%%%%%%%%%%%%%%%%%%%%%%%%%%%%
%\usepackage{setspace}
%\singlespacing        %% 1-spacing (default)
%\onehalfspacing       %% 1,5-spacing
%\doublespacing        %% 2-spacing


%% Other Packages %%%%%%%%%%%%%%%%%%%%%%%%%%%%%%%%%%%%%%%%%%%
%\usepackage{a4wide} %%Smaller margins = more text per page.
%\usepackage{fancyhdr} %%Fancy headings
%\usepackage{longtable} %%For tables, that exceed one page


%%%%%%%%%%%%%%%%%%%%%%%%%%%%%%%%%%%%%%%%%%%%%%%%%%%%%%%%%%%%%
%% Remarks
%%%%%%%%%%%%%%%%%%%%%%%%%%%%%%%%%%%%%%%%%%%%%%%%%%%%%%%%%%%%%
%
% TODO:
% 1. Edit the used packages and their options (see above).
% 2. If you want, add a BibTeX-File to the project
%    (e.g., 'literature.bib').
% 3. Happy TeXing!
%
%%%%%%%%%%%%%%%%%%%%%%%%%%%%%%%%%%%%%%%%%%%%%%%%%%%%%%%%%%%%%

%%%%%%%%%%%%%%%%%%%%%%%%%%%%%%%%%%%%%%%%%%%%%%%%%%%%%%%%%%%%%
%% DOCUMENT
%%%%%%%%%%%%%%%%%%%%%%%%%%%%%%%%%%%%%%%%%%%%%%%%%%%%%%%%%%%%%
\begin{document}

\pagestyle{empty} %No headings for the first pages.


%% Title Page %%%%%%%%%%%%%%%%%%%%%%%%%%%%%%%%%%%%%%%%%%%%%%%
%% ==> Write your text here or include other files.

%% The simple version:
%\title{An Overview of Semantic Segmentation and \\
			 %Pushing State of the Art}
%\author{Chase Gaudet}
%\date{} %%If commented, the current date is used.
%\maketitle

%% The nice version:
%% Based on a TeXnicCenter-Template by Tino Weinkauf.
%%%%%%%%%%%%%%%%%%%%%%%%%%%%%%%%%%%%%%%%%%%%%%%%%%%%%%%%%%%%%

%%%%%%%%%%%%%%%%%%%%%%%%%%%%%%%%%%%%%%%%%%%%%%%%%%%%%%%%%%%%%
%% Deckblatt
%%%%%%%%%%%%%%%%%%%%%%%%%%%%%%%%%%%%%%%%%%%%%%%%%%%%%%%%%%%%%
%%
%% ATTENTION: You need a main file to use this one here.
%%            Use the command "\input{filename}" in your
%%            main file to include this file.
%%

\begin{titlepage}

\begin{center}

%\vspace*{1cm}
\Large
\textsc{Deep Quaternion Networks}\\

\vspace{4cm}

%\LARGE
\textsc{Prospectus\\[0.5\baselineskip]
by\\[0.5\baselineskip]
Chase Gaudet}\\

\vspace{4cm}
\textsc{\today}\\ %%Date - better you write it yourself.

\vspace{1cm}
\textsc{Supervisor:\\
Dr. Anthony Maida}\\

\vspace{1cm}
\textsc{University of Louisiana at Lafayette\\
Faculty of Computer Science}\\

\end{center}

\end{titlepage}
 %%You need a file 'titlepage.tex' for this.
%% ==> TeXnicCenter supplies a possible titlepage file
%% ==> with its templates (File | New from Template...).


%% Front Matter Section %%%%%%%%%%%%%%%%%%%%%%%%%%%%%%%%%%%%%
\chapter*{Abstract}
The field of deep learning has seen significant advancement in recent years.
However, much of the existing work has been focused on real-valued numbers.
Recent work has shown that a deep learning system using the complex numbers can be deeper for a fixed parameter budget compared to its real-valued counterpart.
In this work, we explore the benefits of generalizing one step further into the hyper-complex numbers, quaternions specifically, and provide the architecture components needed to build deep quaternion networks.
We go over quaternion convolutions, present a quaternion weight initialization scheme, and present algorithms for quaternion batch-normalization.
These pieces are tested in a classification model by end-to-end training on the CIFAR-10 and CIFAR-100 data sets and a segmentation model by end-to-end training on the KITTI Road Segmentation data set. 
The quaternion networks show improved convergence compared to real-valued and complex-valued networks, especially on the segmentation task.


\chapter*{Dedication}
PLACEHOLDER


%% Inhaltsverzeichnis %%%%%%%%%%%%%%%%%%%%%%%%%%%%%%%%%%%%%%%
\tableofcontents %Table of contents
\cleardoublepage %The first chapter should start on an odd page.

\pagestyle{plain} %Now display headings: headings / fancy / ...



%% Chapters %%%%%%%%%%%%%%%%%%%%%%%%%%%%%%%%%%%%%%%%%%%%%%%%%
%% ==> Write your text here or include other files.

%\input{intro} %You need a file 'intro.tex' for this.

\chapter{Quaternions}
The quaternions are a number system that extends the complex numbers. 
They were first described by Irish mathematician William Rowan Hamilton in 1843 and applied to mechanics in three-dimensional space. 
In modern mathematical language, quaternions form a four-dimensional associative normed division algebra over the real numbers, and therefore also a domain.
In this chapter we will define operations on the quaternion algebra, draw connection to the complex numbers and $\mathbb{R}^3$, and show some real world use cases of quaternions.


\section{Quaternion Algebra}\label{s:quatalg}
In 1833 Hamilton proposed complex numbers $\mathbb{C}$ be defined as the set $\mathbb{R}^2$ of ordered pairs $(a, b)$ of real numbers.
He then began working to see if triplets $(a,b,c)$ could extend multiplication of complex numbers.
In 1843 he discovered a way to multiply in four dimensions instead of three, but the multiplication lost commutativity.
This construction is now known as quaternions.
Quaternions are composed of four components, one real part, and three imaginary parts.
Typically denoted as
\begin{equation}
\mathbb{H} = \{a + b\textit{i} + c\textit{j} + d\textit{k}~:~a,b,c,d \in \mathbb{R}\}
\label{eq:quaternion1}
\end{equation}
where $a$ is the real part, $(i,j,k)$ denotes the three imaginary axis, and $(b,c,d)$ denotes the three imaginary components.
Sometimes $a$ is referred to as the scalar part and $(a,b,c)$ as the vector part.
Quaternions are governed by the following arithmetic:
\begin{equation}
i^2=j^2=k^2=ijk=-1
\label{eq:quarternion2}
\end{equation}
which, by enforcing distributivity, leads to the noncommutative multiplication rules
\begin{equation}
ij=k,~jk=i,~ki=j,~ji=-k,~kj=-i,~ik=-j
\label{eq:quarternion3}
\end{equation}


\subsection{Addition and Multiplication}
The addition of two quaternions acts component wise, exactly the same as two complex numbers.
Consider the quaternion $q$
\begin{equation}
q = q_0 + q_1\textit{i} + q_2\textit{j} + q_3\textit{k}
\label{eq:q}
\end{equation}
and the quaternion $p$
\begin{equation}
p = p_0 + p_1\textit{i} + p_2\textit{j} + p_3\textit{k}.
\label{eq:p}
\end{equation}
There addition is given by
\begin{equation}
p + q = (p_0+q_0) + (p_1+q_1)\textit{i} + (p_2+q_2)\textit{j} + (p_3+q_3)\textit{k}.
\label{eq:quataddition}
\end{equation}

The product of two quaternions will produce another quaternion and is given by
\begin{align*}
pq &= (p_0 + p_1\textit{i} + p_2\textit{j} + p_3\textit{k})(q_0 + q_1\textit{i} + q_2\textit{j} + q_3\textit{k}) \nonumber \\
&= p_0q_0 - (p_1q_1 + p_2q_2 + p_3q_3) \nonumber \\
&~~~ + p_0(q_1\textit{i} + q_2\textit{j} + q_3\textit{k}) + q_0(p_1\textit{i} + p_2\textit{j} + p_3\textit{k}) \nonumber \\
&~~~ + (p_2q_3 - p_3q_2)\textit{i} + (p_3q_1 - p_1q_3)\textit{j} + (p_1q_2p - p_2q_1)\textit{k}. \nonumber
\end{align*}
This can be greatly simplified by utilizing the inner and cross products of two vectors in $\mathbb{R}^3$:
\begin{equation}
pq = p_0q_0 - \textbf{p}\cdot\textbf{q} + p_0\textbf{q} + q_0\textbf{p} + \textbf{p} \times \textbf{q}
\label{eq:quatmult}
\end{equation}
where $\textbf{p} = (p_1,p_2,p_3)$ and $\textbf{q} = (q_1,q_2,q_3)$ are the vector parts of $p$ and $q$ respectively.


\subsection{Conjugate, Norm, and Inverse}
The \textit{conjugate} of $q$, denoted as $q^*$, is defined as a negation of the vector part of $q$
\begin{equation}
q^* = q_0 - \textbf{q}
\label{eq:quatconjugate}
\end{equation}
and is constructed such that
\begin{equation*}
qq^* = q_0q_0.
\end{equation*}

The \textit{norm} of a quaternion $q$, denoted as $|q|$, is the scalar
\begin{equation*}
|q| = \sqrt{q^*q}
\end{equation*}
and a quaternion is said to be a \textit{unit quaternion} if its norm is 1.

The \textit{inverse} of a quaternion $q$ is defined as 
\begin{equation*}
q^{-1} = \frac{q^*}{|q|^2},
\end{equation*}
which gives $q^{-1}q = qq^{-1} = 1.$
For all unit quaternions, the inverse is equal to the conjugate.


\section{Geometric Representation of Quaternions}
Here we will try to show a more intuitive geometric interpretation of quaternions by showing their link to complex numbers and how they can be used to represent rotations.
To do this we will first give a brief reminder of complex numbers and their geometric interpretation of rotating a 2D plane.

\subsection{Complex Algebra}
Complex numbers are composed of two components, one real part, and one imaginary part.
This is usually denoted as
\begin{equation}
\mathbb{C} = \{a + bi~:~a,b \in \mathbb{R}, ~~i^2=-1\}
\label{eq:complexalgebra}
\end{equation}
where $a$ is the real part, $i$ denotes the single imaginary axis, and $b$ denotes the single imaginary component.

Let $c$ and $d$ be two complex numbers, their addition is given by
\begin{equation}
c + d = (c_0 + c_1i) + (d_0 + d_1i) = (c_0+d_0) + (c_1+d_1)i
\label{eq:complexaddition}
\end{equation}
and their multiplication by
\begin{equation}
cd = (c_0d_0 - c_1d_1) + (c_0d_1 + c_1d_0)i.
\label{eq:complexmult}
\end{equation}
Any complex number has a length given by
\begin{equation}
|c| = |c_0+c_1i| = \sqrt{c_0^2 + c_1^2}
\label{eq:complexlength}
\end{equation}
and like quaternions, any complex number with a length of 1 is called a \textit{unit complex number}.


\subsection{Complex Rotation Operation}
The set of unit complex numbers lies on the unit circle in $\mathbb{C}$ and Leonhard Euler showed that
\begin{equation}
e^{i\theta} = \mbox{cos}~\theta + i~\mbox{sin}~\theta.
\label{eq:euler}
\end{equation}
If we multiply this by any positive number $r$, we get a complex number of length $r$.
Therefore, by adjusting the length $r$ and the angle $\theta$, we can write any complex number.
This form goes by the name \textit{polar coordinates}.

They are a great way to multiply complex numbers. 
Instead of \eqref{eq:complexmult} let us write each complex in polar coordinates
\begin{equation*}
c = (c_0+c_1i) = re^{i\theta}, ~~~d = (d_0+d_1i) = se^{i\phi}
\end{equation*}
and then multiply
\begin{equation*}
cd = re^{i\theta}se^{i\phi} = rse^{i(\theta+\phi)}.
\end{equation*}
This tells us that to multiply two complex numbers, multiply their lengths and add their angles.
In particular, if we multiply a given complex number by a unit complex number the resulting length is the same, but we have rotated it by $\theta$ degrees. 
This gives us some hints that quaternions may also be able to perform rotation operation in higher dimensional space.



\subsection{Quaternion Rotation Operation}
We will stick to interpretations of quaternions in $\mathbb{R}^3$ as it is easier to visualize, but how can a quaternion, which exists in $\mathbb{R}^4$, operate on a 3D vector?
Recall that the imaginary components of a quaternion is called the vector part and a vector $\textbf{v} \in \mathbb{R}^3$ is a \textit{pure quaternion} whose real part is zero.

\begin{figure*}[!h]
	\centering
		\includegraphics[width=1.0\textwidth]{figures/r3.png}
	\caption{$\mathbb{R}^3$ can be viewed as a subspace of quaternions called pure quaternions which have a real part of zero.}
	\label{f:r3}
\end{figure*}

Using the unit quaternion $q$ let us define a function $L_q:~\mathbb{R}^3 \rightarrow \mathbb{R}^3$ on vectors $\textbf{v} \in \mathbb{R}^3
$:
\begin{align}
L_q(\textbf{v}) &= q\textbf{v}q^* \nonumber \\
&= (q_0^2 - ||\textbf{q}||^2)\textbf{v} + 2(\textbf{q}\cdot\textbf{v})\textbf{q} + 2q_0(\textbf{q} \times \textbf{v}). 
\label{eq:Lq}
\end{align}

Two observations to note are that first, the operation \eqref{eq:Lq} does not modify the length of the vector $\textbf{v}$:
\begin{align*}
||L_q(\textbf{v})|| &= ||q\textbf{v}q^*||  \\
&= |q| \cdot ||\textbf{v}|| \cdot |q^*| \\
&= ||\textbf{v}||. 
\end{align*}
And second, the direction of $\textbf{v}$, if along $\textbf{q}$, is left unchanged by the function. To verify let $\textbf{v} = k\textbf{q}$
\begin{align*}
L_q(\textbf{v}) &= q(k\textbf{q})q^*  \\
&= (q_0^2 - ||\textbf{q}||^2)k\textbf{q} + 2(\textbf{q}\cdot k\textbf{q})\textbf{q} + 2q_0(\textbf{q} \times k\textbf{q}) \\
&= k(q_0^2 + ||\textbf{q}||^2)\textbf{q} \\
&= k\textbf{q}.
\end{align*}
Using our insights from complex numbers this lets us guess that the function \eqref{eq:Lq} acts like a rotation about $\textbf{q}$.

\begin{theorem}
For any unit quaternion
\begin{equation}
q = q_0 + \textbf{q} = \mbox{cos}\frac{\theta}{2} + \mathbf{\hat{u}}\mbox{sin}\frac{\theta}{2},
\label{eq:unitquat}
\end{equation}
and for any vector $\textbf{v} \in \mathbb{R}^3$ the result of the function
\begin{equation*}
L_q(\textbf{v}) = q\textbf{v}q^*
\end{equation*}
on $\textbf{v}$ is equivalent to a rotation of the vector through an angle $\theta$ about $\mathbf{\hat{u}}$ as the axis of rotation.
\end{theorem}

\begin{proof}
Given a vector $\textbf{v} \in \mathbb{R}^3$, we decompose it as $\textbf{v} = \textbf{a} + \textbf{n}$, where $\textbf{a}$ is the component along the vector $\textbf{q}$ and $\textbf{n}$ is the component normal to $\textbf{q}$. 
Then we show that under the function $L_q$, $\textbf{a}$ is invariant, while $\textbf{n}$ is rotated about $\textbf{q}$ through an angle $\theta$. 

Earlier we showed that $\textbf{a}$ is invariant under $L_q$ so let us see how $L_q$ transform $\textbf{n}$.
\begin{align*}
L_q(\textbf{n}) &= (q_0^2 - ||\textbf{q}||^2)\textbf{n} + 2(\textbf{q} \cdot \textbf{n})\textbf{q} + 2q_0(\textbf{q} \times \textbf{n}) \\
&= (q_0^2 - ||\textbf{q}||^2)\textbf{n} + 2q_0(\textbf{q} \times \textbf{n}) \\
&= (q_0^2 - ||\textbf{q}||^2)\textbf{n} + 2q_0||\textbf{q}||(\mathbf{\hat{u}} \times \textbf{n}),
\end{align*}
where in the last step we introduced $\mathbf{\hat{u}} = \textbf{q}/||\textbf{q}||$.
Let $\textbf{n}_{\bot} = \mathbf{\hat{u}} \times \textbf{n}$ to get
\begin{equation}
L_q(\textbf{n}) = (q_0^2 -||\textbf{q}||^2)\textbf{n} + 2q_0||\textbf{q}||\textbf{n}_{\bot}.
\label{eq:p11}
\end{equation}
Also note that $\textbf{n}_{\bot}$ and $\textbf{n}$ have the same length:
\begin{equation*}
||\textbf{n}_{\bot}|| = ||\textbf{n} \times \mathbf{\hat{u}}|| = ||\textbf{n}|| \cdot ||\mathbf{\hat{u}}||\mbox{sin}\frac{\pi}{2} = ||\textbf{n}||.
\end{equation*}
Then rewriting \eqref{eq:p11} we arrive at
\begin{align*}
L_q(\textbf{n}) &= \left( \mbox{cos}^2 \frac{\theta}{2} - \mbox{sin}^2 \frac{\theta}{2} \right) \textbf{n} + \left( 2\mbox{cos} \frac{\theta}{2} \mbox{sin} \frac{\theta}{2} \right) \textbf{n}_{\bot} \\
&= \mbox{cos}~\theta \textbf{n} + \mbox{sin}~\theta \textbf{n}_{\bot}.
\end{align*}
The resulting vector is a rotation of $\textbf{n}$ through an angle $\theta$ in the plane defined by $\textbf{n}$ and $\textbf{n}_{\bot}$.
\end{proof}
















\chapter{Introduction To Convolutional Neural Networks}
The Convolutional Neural Network (CNN) is a well-known deep learning architecture that was loosely inspired by the natural visual perception mechanism of some living organisms.
This comes from a paper by Hubel and Wiesel \cite{hubel1968receptive} in 1959 where they found that their were specific cells in the visual cortex of animals that were responsible for detecting light in the receptive field, which eventually won them a Nobel Prize. 
It was not until 1990 that LeCun et al. \cite{lecun1990handwritten} released the crucial paper that would establish modern framework for CNNs. 
Their model was a multiple layer neural network called LeNet-5 that could classify 28x28 greyscale hand written digit images with very high accuracy. 
As seen in Fig.~\ref{f:lenet} LeNet-5 has multiple layers and can be trained with the back-propagation algorithm \cite{hecht1988theory}, but used a low number of parameter values for its size. 
To achieve this LeNet-5 used two core components, which we will discuss in the next section.

\begin{figure}[h!]
	\centering
		\includegraphics[width=0.95\textwidth]{figures/lenet5.png}
	\caption{The architecture for LeNet-5, which is composed of repeating convolution and pooling layers.}
	\label{f:lenet}
\end{figure}

\section{Basic CNN Components}
While there are many different layer types in current literature, here we will focus on the two introduced in LeNet-5, which are still used today along with some variants of them. 

\subsection{Convolutional Layer}
The first is the convolution layer. 
Its purpose is to learn feature representations of the inputs. 
Each convolution layer has a specified number of kernels where each kernel is unique as to generate a distinct feature map. 
Each neuron of a feature map is connected to a region of neighboring neurons in the previous layer. 
The size of the connection is known as the kernel size or the receptive field size. 
The feature map is obtained by first convolving the input with the kernel and then applying an element-wise nonlinear activation function on the result. 
Because it is a convolution operation each kernel is shared with all spatial locations of the input. 
This spatial weight sharing is what creates the low parameter number compared to a regular neural network which fully connects all inputs and outputs. 
It also allows the CNN to take advantage of the underlying structure in images. 
Topological information, i.e., spatial information about the structure in an image, such as adjacency and rotations are also taken into account. 
The equation for finding the feature value at location $(i,j)$ in the $k^{th}$ feature map of the $l^{th}$ layer, $z^l_{i,j,k}$ is
\begin{equation}
z^l_{i,j,k} = {\textbf{w}^l_k}^T \textbf{x}^l_{i,j} + b^l_k
\label{eq:convbasic}
\end{equation}
where $\textbf{w}^l_k$ and $b^l_k$ are the weight vector, or convolution kernel, and the bias term of the $l^{th}$ layer respectively, and $\textbf{x}^l_{i,j}$ is the input patch centered at location $(i,j)$ of the $l^{th}$ layer. 
An example of this operation can be seen in Fig.~\ref{f:convolution}. 
After the convolution is applied the result is then put through the nonlinear activation function:
\begin{equation}
a^l_{i,j,k} = a(z^l_{i,j,k})
\label{eq:activationbasic}
\end{equation}
where $a(\cdot)$ is the nonlinear activation function. 
We will discuss different activation functions in great detail in Chapter \ref{StateOfTheArt}.

\begin{figure}[h!]
	\centering
		\includegraphics[width=0.95\textwidth]{figures/convolution.png}
	\caption{Convolution operation performed on a single feature location and one convolution kernel \cite{convolution}.}
	\label{f:convolution}
\end{figure}


\subsection{Pooling Layer}
The other key layer is the pooling layer shown in Fig.\ref{f:lenet} as subsampling. 
The pooling layer comes after a convolution layer and it performs a reduction in the resolution of the feature map. 
This layer works typically with two arguments: the spatial size of the pooling window $F$ and the stride, or shift per operation, $S$.
It is accomplished in a similar manner to convolution, starting at the top right a window of size $F \times F$ is grabbed.
From this window the pooling function is used, which is commonly a \textbf{MAX} function.
The window then moves over by $S$ pixels and the above operation happens again.
This is repeated until the whole image is covered.
For example, if one chooses $F = 2$ and $S = 2$ they will be selecting the maximum pixel in a $2 \times 2$ window and moving 2 pixels over exactly as shown in Fig.~\ref{f:maxpool}.
Note that this reduces the original image by a factor of $F = 2$.
This serves the purpose of reducing the number of multiplications the architecture must perform and also aims to achieve shift invariance. 
We will discuss different pooling techniques in Chapter \ref{StateOfTheArt}.

\begin{figure}[h!]
	\centering
		\includegraphics[width=0.95\textwidth]{figures/maxpool.png}
	\caption{Example max pooling operation with a size of 2x2. A 2x2 window is moved over the input with a stride of 2 and the maximum value of the window is taken \cite{maxpooling}.}
	\label{f:maxpool}
\end{figure}

There is also the inverse of pooling, often called upsampling, that will increase the resolution of the feature maps.
There are several cases where this is a desired effect which we will discuss later.


\subsection{CNN Common Tasks}
The last layers of LeNet-5 are fully connected layers like that of a regular Feed-Forward Neural Network (FNN), sometimes called Multilayer Perceptrons \cite{rumelhart1985learning}. 
The second to last layer is sometimes called a decision layer as it takes the outputs from the feature mapping and combines them together. 
The last layer is a fully connected layer of size 10, which corresponds to the number of classes in that particular data set. 
The classes, which are digits numbered 0 through 9, are represented as a `one hot' vector, meaning that if the class is the digit `0' the first element of the output vector should be 1 and the rest should be 0. 
This can be thought of as a probability distribution on the classes. 
This is an important representation because it allows one to use loss functions that minimizes the differences of distributions opening up many potential loss functions.
This was the common use case for CNNs for a while, given an image predict a distribution over a set of classes. 
This task is known as \textit{classification}. 

Recently CNNs have seen use in another area known as \textit{semantic segmentation}.
Semantic segmentation, also sometimes just called segmentation, is the process of partitioning an image into multiple segments, or sets of pixels. 
More precisely, segmentation is the process of assigning a class from our classification set to each pixel, or potentially area of pixels, in an image. 
Unlike classification, the output of this CNN would typically be in the form of an image.
Typically the network will output a distribution over a set of classes \textit{per pixel} and then the pixel is labeled with the class with the highest probability. 
There are many applications where image classification alone does not giving enough information and one needs pixel-level labels. 
Examples include: detecting road signs \cite{maldonado2007road}, detecting tumors in medical imaging \cite{li2015automatic, lyksborg2015ensemble, kainz2015semantic, havaei2017brain}, detecting objects of interest from satellite photos \cite{chen2013vehicle}, finding pedestrians \cite{du2016fused}, and many more. 
Having pixel level labels allows multiple objects of different classes to be detected in one image compared to image level labels where there is a single output per image.

In recent years, CNNs have obtained state of the art results on almost all classification and segmentation data sets. 
The improvements in CNNs have come from a few different areas including better computers, improvements to initialization of network weights, and specific architectures to combat problem areas of deep networks. 
In this next chapter we will go through a list of improvements and key papers.  
\chapter{State of the Art}\label{StateOfTheArt}
In this chapter we will go over the latest improvements for CNNs. 
They range from the activation function used within each neuron to major architecture changes to combat problems that arise from training deeper and deeper networks.


\section{Activation Functions}
The current trend has been using activation functions that are non-saturated, meaning they do not have very small derivatives at large input values. 
Also, these new functions are chosen to help combat the `vanishing gradient problem' \cite{hochreiter2001gradient}.
This problem was due to many of the originally used activation functions (e.g sigmoid or tanh) 'squash' their input into a very small output range in a very non-linear fashion. 
For example, sigmoid maps the real number line onto a `small' range of $[0, 1]$. 
As a result, there are large regions of the input space which are mapped to an extremely small range. 
In these regions of the input space, even a large change in the input will produce a small change in the output - hence the gradient is small.
This becomes much worse when we stack multiple layers of such non-linearities on top of each other, leading to difficulty in training deeper networks.
We will go through a few new activation functions, whose plots are shown in Fig.~\ref{f:activations}.

\subsection{ReLU:}
The most notable non-saturated activation function is the rectified linear unit (ReLU) \cite{nair2010rectified}. 
This is defined as:
\begin{equation}
f(x) = \mbox{max}(0,x).
\label{e:relu}
\end{equation}
The benefits of this unit include: faster calculation since the max($\cdot$) operation is faster than sigmoid or tanh, it creates sparsity in the hidden units by giving true zero values often to help sparse representations form, and does not suffer from vanishing gradients in deep models. 
ReLU has been shown to work better than sigmoid or tanh in several tasks and shows fast convergence even without pretraining \cite{glorot2011deep,krizhevsky2012imagenet,zeiler2013rectified,maas2013rectifier}.

\subsection{LReLU:}
The ReLU has zero gradient whenever its inputs add to a negative value or when a large gradient changes the weights to large negatives value making future input into the unit always have a negative value. 
These units will never be updated and learning will not occur, which can be a problem. 
Mass \textit{et al}. 
\cite{maas2013rectifier} found a solution to this by introducing a non zero component to the ReLU when the input is negative. 
They called this unit the Leaky ReLU as it 'leaks' a positive gradient on the negative side of its graph by having a very small slope. 
This leaky ReLU or LReLU is defined as:
\begin{equation}
f(x) = \mbox{max}(0,x) + \lambda ~\mbox{min}(0,x)
\label{e:lrelu}
\end{equation}
where $\lambda \in (0,1)$ and is predefined per model. 
This unit allows for small gradients when the unit is not active, which helps the problems of the ReLU unit.

\subsection{PReLU:}
Rather than trying to find an optimal value for $\lambda$ in the LReLU, He et al. \cite{he2015delving} introduced the Parametric ReLU which adapts the $\lambda$ during training. 
It is defined as:
\begin{equation}
f(x) = \mbox{max}(0,x) + \lambda_k ~\mbox{min}(0,x)
\label{e:prelu}
\end{equation}
where $\lambda_k$ is the $k$-th channel. 
Since the number of parameters in networks is often very large compared to the number of total channels, the extra computational cost to learn the values of $\lambda_k$'s is not much.

\subsection{ELU:}
Next is the Exponential Linear Unit (ELU) \cite{clevert2015fast}. 
ELUs are similar to the above mentioned units, but they have a saturating function on their negative side. 
The saturation function decreases the variation of the units if they are not activated, which gives those units a chance to update while making them more robust to noise. 
The ELU function is defined as:
\begin{equation}
f(x) = \mbox{max}(0,x) + \mbox{min}(0,\lambda(e^{x}-1))
\label{e:elu}
\end{equation}
where $\lambda$ is predefined as in the LReLU.

\subsection{PELU:}
A natural extension of the ELU is the Parametric ELU \cite{trottier2016parametric}. 
Unlike the PReLU which learns one parameter to modify the LReLU, the PELU learns two parameters during training to modify the ELU. 
This gives the network more control over the vanishing gradients. 
The PELU is defined as:
\begin{equation}
f(x) = \mbox{max}(0,\frac{a}{b}x) + \mbox{min}(0,a(e^{x/b}-1))
\label{e:pelu}
\end{equation}
where $a,b > 0$. 
Both $a$ and $b$ change the slope of the linear function together, $b$ effects the scale of the exponential decay, and $a$ is the saturation point in the negative side of the function. 
In \cite{trottier2016parametric} it is shown that PELU does a better job than all the above functions in several different models and data sets.

\begin{figure}[h!]
	\centering
		\includegraphics[width=0.85\textwidth]{figures/activations.png}
	\caption{Examples of some of the discussed activation functions. From top to bottom and left to right: ReLU, LReLU, PReLU, and ELU.}
	\label{f:activations}
\end{figure}


\section{Weight Initialization}
The parameters of a network are randomly initialized because if they were not, there would exist symmetry among the parameters and nothing useful would be learned.
The deeper and more complex the network, the more parameters one is attempting to optimize.
Each layer of the network will scale its input by some constant $k$, meaning the final layer will scale the original input by $L^k$ where $L$ is the number of layers.
It is clear that for deep network that values $k > 1$ will have very large outputs, while values $k < 1$ lead to outputs to diminish to zero.
Because of this determining an optimal way to initialize these parameters was vital to the network converging to a good set of weights.
There are two commonly used schemes we will mention in this chapter, Glorot and Bengio \cite{glorot2010understanding} and a modification to that by He et al. \cite{he2015delving}.

\subsection{Glorot Initialization}
A big assumption in the derivation of this initialization scheme is that the neurons are linear functions.
Let $X$ be the input to $n$ linear neurons with weights $W$ that produce output $Y$
\begin{equation*}
Y = W_1X_1 + W_2X_2 + ~...~ + W_nX_n.
\end{equation*}
To find the variance of $Y$ first observe that
\begin{equation*}
\mbox{Var}(W_iX_i) = \mbox{Var}(W_i)\mbox{Var}(X_i)
\end{equation*}
and if assumed that $X_i$ and $W_i$ are all independently and identically distributed one can write $Y$ as
\begin{equation*}
\mbox{Var}(Y) = \mbox{Var}(W_1X_1 + W_2X_2 + ~...~ + W_n+X_n) = n\mbox{Var}(W_i)\mbox{Var}(X_i).
\end{equation*}
This tells us under these assumptions that the variance of the input is the variance of the input scaled by $n\mbox{Var}(W_i)$, so if we want the variance input and output to be the same the variance of the weights should be
\begin{equation*}
\mbox{Var}(W_i) = \frac{1}{n_{in}}
\end{equation*}
where we introduce the subscript on $n$ to denote that this is the count of neurons going into the layer.
The reason for that is because the above must also be done for the backpropagated pass where the same formula is found except $n$ is the number of neurons going out of the layer
\begin{equation*}
\mbox{Var}(W_i) = \frac{1}{n_{out}}.
\end{equation*}
These two constrains can only be satisfied if $n_in = n_out$, so Glorot and Bengio take the average
\begin{equation}
\mbox{Var}(W_i) = \frac{2}{n_{in} + n_{out}}.
\label{eq:glorotinit}
\end{equation}
In practice it turns out that the backpropagation pass is less important to preserve compared to the forward pass.
Because of this and the fact that finding $n_{out}$ is expensive to find in most software implementations the Glorot initialization is often shortened to
\begin{equation}
\mbox{Var}(W_i) = \frac{1}{n_{in}}.
\label{eq:glorotshort}
\end{equation}


\subsection{He Initialization}
He et al. set out to define a good initialization solely for the ReLU activation function.
They build on Glorot and Bengio's work by noting that the ReLU is almost a linear activation function, but is zero for half of its input, so you need to double the size of weight variance to keep the signal’s variance constant.
This leads to 
\begin{equation}
\mbox{Var}(W_i) = \frac{2}{n_{in}}.
\label{eq:heinit}
\end{equation}
It is of note that using this initialization scheme they were the first to surpass human level performance on the ImageNet 2012 classification challenge.


\section{Regularization}
Some of the biggest improvements to NNs have been through regularizing the network in different ways to prevent overfitting.
Overfitting means that the network learns the training set well, but performs poorly on any new or unseen data.

\subsection{Dropout}
Introduced by Hinton \textit{et al.} \cite{hinton2012improving}, Dropout has shown to be very effective at improving network results and reducing overfitting. 
In their paper Dropout is applied to fully-connected layers where the output of the Dropout layer is 
\begin{equation}
\textbf{y} = \textbf{r} \ast a(\textbf{W}^T\textbf{x}), 
\label{eq:dropout}
\end{equation}
where $\textbf{x} = [x_1,x_2,...,x_n]^T$ is the input to the fully-connected layer, $a(\cdot)$ is an activation function, $\textbf{W}$ is the weight matrix, $\ast$ is an element-wise multiplication, and $\textbf{r}$ is a binary vector whose elements are independently drawn from a Bernoulli distribution with parameter $p$. 
This means that at every update to the network, each neuron has a $p$ chance of getting multiplied by zero. 
Dropout usually prevents the network from becoming too reliant on a few neurons and helps the network generalize by spreading feature information. 
Some extensions to Dropout have been proposed like in \cite{wang2013fast} where a method to improve the speed of training while using Dropout are introduced. 
Similar to the learned parameters of the activation functions, \cite{ba2013adaptive} learn the $p$ in the Dropout distribution adaptively. 
Finally, in \cite{tompson2015efficient} they modify Dropout with CNNs specifically in mind by extending the Dropout value across the entire feature map, meaning that during training each feature map has a $p$ chance of being completely ignored for each update step. 
They call this method Spatial Dropout and it seems to help the network learn more general features, which improves results on limited size data sets.

\subsection{Batch Normalization}
It is standard practice to normalize data to have zero-mean and unit variance before feeding it into a NN, but as the data goes through the network, especially deep networks, the data will lose this property. 
This change to the input distribution is known as internal covariance shift. 
To keep data normed through the network \cite{ioffe2015batch} introduced an efficient method called Batch Normalization (BN). 
BN works by normalizing the mean and variance of each layers input using each batch rather than the entire training set. 
To demonstrate BN, let $\textbf{x} = [x_1,x_2,...,x_n]^T$ be a $n$ dimensional input to a layer. 
The $k$-th dimensions is normalized by:
\begin{equation}
\hat{x}_k = (x_k - \mu_{B})/\sqrt{\sigma^2_B + \epsilon}
\label{eq:BN1}
\end{equation}
where $\mu_{B}$ and $\sigma^2_B$ are the mean and variance of the batch respectively and $\epsilon$ is some small constant value that will guarantee the root term is always defined. 
Finally the input $\hat{x}_k$ is further transformed:
\begin{equation}
y_k = \mbox{BN}_{\alpha,\beta}(x_k) = \alpha \hat{x}_k + \beta
\label{eq:BN2}
\end{equation}
where $\alpha$ and $\beta$ are parameters learned during training. 
There are many benefits to using BN. 
By reducing internal covariance shift the network will converge faster. 
By making sure no inputs get too large or small BN makes it possible to use saturating activation functions without fear of getting stuck in a saturating section and having vanishing gradients.

\section{Architecture Improvements}
There are a couple of recent architecture improvements to NNs that both help with speed (by lowering computational cost for similar results) and immensely with vanishing gradients.

\subsection{Network-In-Network} 
The first was introduced in \cite{lin2013network} where they use fully connected layers on the outputs of each convolution operation called Network-in-network. 
The idea was an alternative to stacking multiple convolutional layers (each with a number of their own filters) to get deepness in a neural network model, by replacing each filter with a multi-layer perceptron, which is essentially a small neural network that slides across the image like a convolutional filter. 
The math works out so that a $1\times 1\times U$ convolution filter convolved across a $V$-channel image emulates a $U\times V$ matrix multiplied by each $V$-channel pixel, which is the same as running a single-layer neural network across every pixel of your input as if each pixel were an example vector in a training set. 
Chaining together such $1\times 1\times U$ convolutions, you get the same result as if running a many-layered neural network at each input pixel of $V$ channels. 
The idea from the paper was to turn a convolution filter which is a generalized linear model, into a non-linear model. 
It allows the network to combine channels from previous layer in a non-linear fashion, which can lead to more advanced features.

\subsection{Inception Blocks}
Google's team was inspired by the Network-in-network paper and saw the power of using the 1x1 convolutions not only for non-linearity, but to reduce computational burden of deep models by using the cross-channel pooling aspect to reduce their feature maps before convolution layers. 
In \cite{szegedy2015going} Christian Szegedy and his team introduced the Inception module as seen in Fig.\ref{f:inceptionblock}. 
\begin{figure}[h!]
	\centering
		\includegraphics[width=1.0\textwidth]{figures/inceptionblock.png}
	\caption{An inception block from \cite{szegedy2015going}. Notice the 1x1 convolutions before the 3x3 and 5x5 convolutions are reducing the number of feature channels. This reduces the number of multiplications that must be performed.}
	\label{f:inceptionblock}
\end{figure}

\subsection{Residual Connections}
Another work that focused on fixing the vanishing gradient problem of deep networks was Residual Nets (ResNets) \cite{he2015deep} where they used what is now called shortcut connections. 
These connections were inspired by Long Short Term Memory (LSTM) units, a type of recurrent neural network unit that feeds information back into itself with weighted gates. 
Instead of using weighted gates shortcut connections pass information with the identify function so it is untransformed. 
This means that the activation of deep units can be written as the sum of the activation of some shallower unit and a residual function, which is a series of network layers. 
Or to put simply, the input to a layer group is added to the output of that layer group. 
This is called a Residual Block and can be seen in Fig.\ref{f:resblock}. 
By stacking these residual blocks together one gets a fully residual model. 
This design allows the gradients to be directly propagated to shallower units allowing for networks of 100s of layers to be trained. 
\begin{figure}[h!]
	\centering
		\includegraphics[width=0.80\textwidth]{figures/resblock.png}
	\caption{A residual block from \cite{he2015deep}.}
	\label{f:resblock}
\end{figure}


\section{Conclusion}
In this chapter we covered many advanced concepts of CNNs that have led to large improvements in their performance.
The big areas that have seen improvement are the activation functions, weight initializations, regularization, and overall architecture improvements.
There are other areas to possibly find improvements however.
At present, the vast majority of all work done in neural networks is based on real-valued operations and representations.
There have been some recent work exploring the use of complex numbers and very few the use of hyper-complex number (quaternions) showing promising results due to these numbers having greater representational power to the reals.
In \cite{trabelsi2017deep} they present the first complex forms of complex batch-normalization and complex weight initialization strategies to use along with complex convolution to create a fully complex deep convolutional network.
They achieve some state-of-the-art results over the real-valued counter part models, which leads to the question of creating forms for quaternion batch-normalzation and weight initialization for potentially better results.
In the next chapter we will give the reader a basic foundation of the quaternion algebra to build up to the theory of these quaternion network layers.

\chapter{Work Completed}
The goal is to find which combination of new techniques provides the greatest overall segmentation results. To that end we have taken a particularly challenging data set, the 2D EM segmentation challenge \cite{emdata}. The training data is a set of 30 sections from a serial section Transmission Electron Microscopy (ssTEM) data set of the Drosophila first instar larva ventral nerve cord (VNC). The microcube measures 2 x 2 x 1.5 microns approx., with a resolution of 4x4x50 nm$/$pixel. The corresponding binary labels are provided in an in-out fashion, i.e. white for the pixels of segmented objects and black for the rest of pixels (which correspond mostly to membranes). This data set has many fine lines to segment, is heavily biased towards the membrane class, and has few training examples. The test set for the model does not have ground truth labels released to the public and instead one must submit their predictions for an accuracy measure. Because of this our result figures can only show our predictions and no ground truth label to compare next to it.

\section{Models}
Here we describe the models we have used for results so far. The overall basic shape of the model is shown in Fig.~\ref{f:model} and the individual components are shown in Fig.~\ref{f:modelpieces}. The overall shape was inspired by the U-Net model from Section~\ref{s:unet}. Instead of using any type of pooling layers we instead use a strided convolution to down sample feature maps. Also we differ from the U-Net model in the short connections. In U-Net the feature maps forwarded by the short connections are concatenated, but in our model they are summed. Along with the residual connections inside each ResBlock, these long residual connections make our model a fully residual model. The full residual model is shown to help information flow within and across levels in the network \cite{quan2016fusionnet}. Also by not concatenating we reduce the number of feature maps in the expanding path of the model. For the multi-scale residual block we hope to increase the models spatial contiguity by giving it multiple fields of view around each local point.

The first half of the network is called the encoding path. Along this path the number of feature maps is doubled after each downsampling. After the center ResBlock begins the decoding path. In the decoding part, the number of feature maps is halved per level to maintain the network symmetry. The downsampling and upsampling are done with convolution layers. These convolutional layers serve as a connector to bridge the input feature maps and the residual block because the number of feature maps from the previous layer may differ from that of the residual block. The proposed network performs an end-to-end segmentation from the input data to the final prediction of the segmentation. We train the network with image set pairs corresponding to the image and its segmentation label, compare the output with manual segmentation, and use a loss function to back-propagate to adjust the weights of the network.

\begin{figure}[h!]
	\centering
		\includegraphics[width=0.4\textwidth]{figures/mymodel.png}
	\caption{The model architecture. Feature maps from the down path get summed with feature maps of the corresponding size in the up path. The ResBlock types are shown in Figs.~\ref{f:basic}~\&~\ref{f:multi} and the DConvolution and UConvolution are shown in Fig.~\ref{f:dconv} and Fig.~\ref{f:uconv} respectively.}
	\label{f:model}
\end{figure}
\begin{figure}
    \centering
    \begin{subfigure}[b]{0.39\textwidth}
        \centering
        \includegraphics[width=\textwidth]{figures/basic.png}
        \caption{A basic residual block.}
				\label{f:basic}
    \end{subfigure}
    \hfill
		\begin{subfigure}[b]{0.59\textwidth}
        \centering
        \includegraphics[width=\textwidth]{figures/multiscale.png}
				\caption{A multi-scale residual block.}
        \label{f:multi}
    \end{subfigure}
    \\
    \begin{subfigure}[b]{0.25\textwidth}
        \centering
        \includegraphics[width=\textwidth]{figures/dconv.png}
				\caption{A down sampling block.}
        \label{f:dconv}
    \end{subfigure}
    \hfill
    \begin{subfigure}[b]{0.25\textwidth}
        \centering
        \includegraphics[width=\textwidth]{figures/uconv.png}
				\caption{An upsampling block.}
        \label{f:uconv}
    \end{subfigure}
    \caption{Model components.}
    \label{f:modelpieces}
\end{figure}

\section{Training}
For training we select an image size $s$ and a batch size $B$. We then randomly sample a $s\times s$ image from a randomly selected training image. That sub-image is then modified by first adding a very small amount of Gaussian noise, randomly flipped left to right or top to bottom, and randomly rotated a small angle between -20 and 20 degrees. Sub-images are grabbed in this manner until there are $B$ of them. The batch is then run through the model and the weights are updated. We have a set number of batches that we called an epoch, but each epoch is a different set of batches since all augmentation happens online. After each epoch we observe the model's loss on the validation set and keep a record of it along with the corresponding weights. After the total number of epochs is done running we restore the weights to the point of lowest validation loss.

\section{Experimental Setup}
The proposed deep network is implemented using Keras open-source deep learning library \cite{chollet2015}. This library provides an easy-to-use high-level programming API written in Python, and Theano or TensorFlow can be chosen for a backend deep learning engine. Since the work was done on a Windows PC the Theano backend was chosen as TensorFlow did not support GPU processing on Windows. Training and deployment of the network is conducted on a PC equipped with an Intel i7 CPU with a 64 GB main memory and an NVIDIA GTX Geforce 980 Ti GPU.

\section{Results}
This section will show comparison results from many different models so we will break this section into subsections and begin each subsection with a description of the models being compared. 

\subsection{Basic Comparison}
In this section we are establishing a baseline for each of the three main models. We are comparing: the U-Net \cite{ronneberger2015u}, our model shown in Fig.~\ref{f:model} with the basic residual block shown in Fig.~\ref{f:basic}, and our model shown in Fig.~\ref{f:model} with the multi-scale residual block shown in Fig.~\ref{f:multi}. This baseline is using the most common non-saturating activation function, the ReLU shown in Eq.~\ref{e:relu}. Also we did not apply Batch Normalization in any of these baseline models. The loss function used was a binary cross-entropy on each pixel of the output. The loss plots are shown in Fig.~\ref{f:baseline-loss} and a sample segmentation is shown in Fig.~\ref{f:baseline}.

\begin{figure}[h!]
	\centering
		\includegraphics[width=1.0\textwidth]{figures/baseline_loss.png}
	\caption{The loss and validation loss during training for the baseline models.}
	\label{f:baseline-loss}
\end{figure}
\begin{figure}[h!]
	\centering
		\includegraphics[width=1.0\textwidth]{figures/baseline.png}
	\caption{Sample segmentation on a test image for the baseline models. The test image is shown in its normalize stated, which is scaling the values between minus one and one. From left to right: test image, U-Net segmentation, our model, and our model with multi-scale convolution. The segmentations are shown in a red to blue color map where red is 0 and blue is 1.}
	\label{f:baseline}
\end{figure}

\subsection{Applying Batch Normalization}
Most state of the art models use Batch Normalization. We apply it directly before every non-linearity in the model. This is to keep the values, which can become large from the addition operations of the residual connections, stay mean centered. The loss plots are shown in Fig.~\ref{f:batchnorm-loss} and a sample segmentation is shown in Fig.~\ref{f:batchnorm}. Compare how smooth the losses are compared to the baseline model's loss in Fig.~\ref{f:baseline-loss}. The resulting segmentations also come out much more crisp, especially around the edges of the membranes compared to the basic model's segmentation in Fig.~\ref{f:baseline}. Another interesting observation is that the multi-scale segmentation completely segmented the large membrane structure in the bottom right, while the other two models left the center open. This shows that the wider field of view helped the multi-scale model in this instance.

\begin{figure}[h!]
	\centering
		\includegraphics[width=1.0\textwidth]{figures/batchnorm_loss.png}
	\caption{The loss and validation loss during training for the batchnorm models.}
	\label{f:batchnorm-loss}
\end{figure}
\begin{figure}[h!]
	\centering
		\includegraphics[width=1.0\textwidth]{figures/batchnorm.png}
	\caption{Sample segmentation on a test image for the batchnorm models. From left to right: test image, U-Net segmentation, our model, and our model with multi-scale convolution.}
	\label{f:batchnorm}
\end{figure}

\chapter{Additional Research}
In the above we have explored the effects of combining several new breakthroughs in deep neural networks for the task of semantic segmentation. Going forward we would like to explore using GAN models using the newly proposed Wasserstein and the least square loss functions to combat the training instabilities often seen in GANs. This would be the first to our knowledge implementation of GANs for segmentation with those loss functions. We would use the models we have already explored as the generators for these GANs and could publish a stand alone paper on these results.

We also want to look at the results of running a RNN refinement type model on the outputs of all previous models we have explored as well as the GANs outputs. While this may improve our GANs results, we do not believe it would be enough for its own paper.

We also intend to run all of models on more data sets. For the second data set we chose the Larval Zebrafish EM data, which is another neuronal segmentation challenge. However, this set has released the ground truth labels for the test data. And lastly we want to run on a more natural image challenge so we chose the COCO 2016 Detection Challenge segmentation challenge. The COCO data set has more than 200,000 images and 80 object categories.


\chapter{Timeline}
The additional research to be carried out could be finished over the summer months. The code base is written in a very modular way allowing changes to any part of the models with a single variable change. 

\section{Refinement RNN}
The code for the refinement RNN is already written, but no results have been generated yet. The biggest challenge with this model is that it requires a large amount of GPU memory and it takes a long time to train compared to the other models in this paper. To combat this we intend to deploy this model on AWS instances so it does not tie up my main machine from other work. I am very familiar with using AWS so the setup time will be very minimal.

\section{GAN Model}
The majority of the time will come from the implementation of the GAN model. For only this model we will need to add some new code that handles training the discriminator and generator in simultaneous fashion. We also must implement the proposed loss functions since they are not build into any existing deep learning framework. We plan to attack this problem in steps. First we will implement a simple GAN to generate MNIST examples in a too be determined deep learning framework. Once this is working we will create the new loss functions and test them on the MNIST GAN. After we are confident in our ability to run the simultaneous training steps and that our custom loss functions are working we will implement the segmentation GAN.


%%%%%%%%%%%%%%%%%%%%%%%%%%%%%%%%%%%%%%%%%%%%%%%%%%%%%%%%%%%%%



%%%%%%%%%%%%%%%%%%%%%%%%%%%%%%%%%%%%%%%%%%%%%%%%%%%%%%%%%%%%%
%% BIBLIOGRAPHY AND OTHER LISTS
%%%%%%%%%%%%%%%%%%%%%%%%%%%%%%%%%%%%%%%%%%%%%%%%%%%%%%%%%%%%%
%% A small distance to the other stuff in the table of contents (toc)
\addtocontents{toc}{\protect\vspace*{\baselineskip}}

%% The Bibliography
%% ==> You need a file 'literature.bib' for this.
%% ==> You need to run BibTeX for this (Project | Properties... | Uses BibTeX)
\addcontentsline{toc}{chapter}{Bibliography} %'Bibliography' into toc
\bibliographystyle{alpha} %Style of Bibliography: plain / apalike / amsalpha / ...
\bibliography{bib} %You need a file 'bib.bib' for this.

%% The List of Figures
\clearpage
\addcontentsline{toc}{chapter}{List of Figures}
\listoffigures

%% The List of Tables
\clearpage
\addcontentsline{toc}{chapter}{List of Tables}
\listoftables


%%%%%%%%%%%%%%%%%%%%%%%%%%%%%%%%%%%%%%%%%%%%%%%%%%%%%%%%%%%%%
%% APPENDICES
%%%%%%%%%%%%%%%%%%%%%%%%%%%%%%%%%%%%%%%%%%%%%%%%%%%%%%%%%%%%%
\appendix
%\input{FileName} %You need a file 'FileName.tex' for this.


\end{document}

