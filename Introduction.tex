\chapter{Introduction}


\section{Deep Learning}
In recent years the area of artificial intelligence has seen massive growth.
The majority of work has been around a fairly new area sub-field called Deep Learning (DL).
Deep Learning is broadly described as Neural Networks composed of multiple layers.
One type of DL model called Convolutional Neural Networks (CNNs) has achieved state-of-the-art results in almost all Computer Vision (CV) problems, as well as some outside of CV.
The power of CNNs comes from automatically finding a set of highly expressive, hierarchical features.
As we will discuss more in the CNN section, the way they work is closely related to the way the biological vision system works.


\section{Beyond Real Numbers}
Historically the majority of work done in DL, and even Neural Networks, has been using real-valued numbers.
Recently there has been increased attention on using complex numbers.
Complex numbers have several advantages from a computational, biological, and signal processing perspective \cite{trabelsi2017deep}.
One big motivation is that a complex number holds more information than a real number, complex-valued formulation captures amplitude and phase by definition.
Biologically this can be thought of as the neuron�s  firing rate and the relative timing of its activity.

Quaternions are an extension of the complex numbers.
While the complex plane is 2D, the quaternion space is 4D.
They offer some important benefits that over the complex numbers, but note that since a quaternion can be expressed as a $2\times2$ matrix of complex numbers that we are not giving anything up.
We will show that the main advantages come from a quaternion being able to very compactly represent rotations in 3D space and, more importantly, from the algebraic structure inducing promising properties for CNNs.



\section{Outline}
In the following chapters we will fully cover CNNs and their latest developments as well as give the reader a primer on quaternions.
This sets up necessary background knowledge for us to describe our main contributions, which include a novel quaternion weight initialization scheme and novel algorithms for quaternion batch-normalization.
These pieces, along with quaternion convolutions, allow for the construction of a quaternion CNN (QCNN).
We perform sanity checks on the QCNN by comparing its performance against real-valued and complex CNNs on common benchmark data sets in both classification and segmentation and find the QCNN outperforms both.
With these promising results are goal is to explore QCNN performance on more demanding tasks, particularly ones we believe the algebraic structure of quaternions lends itself too.
